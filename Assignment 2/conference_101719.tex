\documentclass[conference]{IEEEtran}
\IEEEoverridecommandlockouts
% The preceding line is only needed to identify funding in the first footnote. If that is unneeded, please comment it out.
\usepackage{cite}
\usepackage{amsmath,amssymb,amsfonts}
\usepackage{algorithmic}
\usepackage{graphicx}
\usepackage{textcomp}
\usepackage{xcolor}
\def\SPSB#1#2{\rlap{\textsuperscript{\textcolor{red}{#1}}}}
\usepackage{amsmath}
\usepackage{fourier}
\usepackage{hyperref}
\pagenumbering{arabic}

\newcommand\tab[1][0.8cm]{\hspace*{#1}}
\def\BibTeX{{\rm B\kern-.05em{\sc i\kern-.025em b}\kern-.08em
    T\kern-.1667em\lower.7ex\hbox{E}\kern-.125emX}}

\makeatletter

%%%%%%%%%%%for copyright notice
\def\ps@IEEEtitlepagestyle{%
	\def\@oddfoot{\mycopyrightnotice}%
	\def\@evenfoot{}%
}
\def\mycopyrightnotice{%
	{\footnotesize978-1-5386-7902-9/19/\$31.00 \textcopyright2019 IEEE\hfill }
	\gdef\mycopyrightnotice{}
}
%%%%%%%%%%%

\let\old@ps@IEEEtitlepagestyle\ps@IEEEtitlepagestyle
\def\confheader#1{%
	% for the first page
	\def\ps@IEEEtitlepagestyle{%
		\old@ps@IEEEtitlepagestyle%
		\def\@oddhead{\strut\hfill#1\hfill\strut}%
		\def\@evenhead{\strut\hfill#1\hfill\strut}%
	}%
	\ps@headings%
}
\makeatother

\confheader{%
	\parbox{20cm}{2019 11th International Conference Communication Systems \& Networks (COMSNETS)}
}




\begin{document}

\title{Optimal VM Coalition for Multi-Tier Applications
	Over Multi-Cloud Broker Environments\\

{\footnotesize \textsuperscript{}}
}

\author{\IEEEauthorblockN{ Sourav Kanti Addya}
\IEEEauthorblockN{{IIT Kharagpur,India} \\
\IEEEauthorblockA{kanti.sourav@ieee.org}}
\and
\IEEEauthorblockN{Anurag Satpathy}
\IEEEauthorblockA{{NIT Rourkela,India} \\
\IEEEauthorblockA{anurag.satpathy@gmail.com}}
\and
\IEEEauthorblockN{ Sandip Chakraborty}
\IEEEauthorblockA{{IIT Kharagpur, India} \\
\IEEEauthorblockA{sandipc@cse.iitkgp.ac.in}}
\and
\IEEEauthorblockN{\textsuperscript{}Soumya K Ghosh}
\IEEEauthorblockA{{IIT Kharagpur, India} \\
\IEEEauthorblockA{skg@cse.iitkgp.ac.in} }
}

\maketitle
%\thispagestyle{empty} %--to make title page number less
\pagestyle{plain} 
\begin{abstract}
In the current cloud market tradition, multi-cloud
broker environments have an important role to support multi-
tier cloud applications where a set of interconnected virtual
machines (VMs) are required for the deployment and execution
of the application. There are several cloud service providers (SP)
who provide services in the form of different VM instances. For
different SPs, pricing of different configurations is different based
on various technical, service level as well as economic factors.
Also, these SPs have multiple data centers located at different
geographical locations with diverse resources and pricing schemes
due to differences in electricity costs, maintenance costs etc. in
different regions. Considering these facts, in this paper, we pro-
pose a multi-cloud broker environment to select an optimal VM
coalition for multi-tier applications from an SP with minimum
coalition pricing and better quality of service. We model the
problem as a bi-objective optimization problem and solve it using
ant-colony optimization based technique. The entire interaction
of SPs is modeled using game theory and results from simulation
show promising behavior in achieving the trade-off between the
aggregate cost and the QoS.
\end{abstract}

\begin{IEEEkeywords}
Cloud Computing; Data center; Virtual machine
\end{IEEEkeywords}

\section{Introduction}

\tab Cloud computing enables users to host their applications
on the infrastructure provided by various third parties called
as service providers (SPs). Virtualization enables SPs to host
multiple applications in the form of virtual machines (VM)
instances on the same host machine. In general, applications
hosted on cloud are multi-tier in nature [1], [2]. As an
illustration, a typical multi-tier web application consist of three
tiers namely (i.) presentation layer (web tier), (ii.) business
logic layer (App tier) and the data layer (database tier) [3]. In
a typical end-to-end deployment, different layers are deployed
over different VMs, with disparate configurations and variable
memory access patterns. These VMs often communicate with
each other and serve user requests as shown in Fig. 1.\newline
\tab
The pricing policies used by SPs vary with respect to
instance type, location and electricity price of the location
hosting the VM [4]. For instance, pricing policies of some
popular SPs, i.e., Amazon EC2, Microsoft Azure and Google
is listed in Table I (as of November 22, 2018). It can be
observed form the table that prices of different instance types
are different. It can also be observed that prices for a same
instance type for same/different SPs are different based on the
location of the DC hosting such VMs. Hence, a user is left
with multiple options to choose from. To relive users from the agony of 
\begin{figure}[htbp]
	\centerline{\includegraphics[height=40mm,width=90mm]{fig0.png}}
	\caption{Multi-tier Applications: Every tier is deployed on a different VM and
		the VM communicates with each other to provide end-to-end services.}
	\label{fig}
\end{figure}
selection, we, in this paper, introduce a centralized
entity called as cloud-broker (CB) that performs this activity
on behalf of users. In fact, a CB makes best possible selection
of VMs from an underlying SP to host a set of VMs supporting
the user application. A request to host a service is received at
the CB and is processed at the queue. The CB has many SPs
at its disposal, and it tries to find the best possible service set
(called the coalition of VMs or the set of VMs) such that the
aggregate price of hosting the application is minimum.
\newline
\tab In this paper, we model the environment using game theory
which is a popular approach to model the strategic interactions
among multiple competing players. The outcome of a game
depends on the actions of other players. The SPs compete with
each other to maximize their individual profit by maximizing
the number of requests accepted. At the same time, the aggregated price of hosting an application is kept at an affordable
rate to attract more applications. This makes game theory
an apt framework to model the strategic interactions among
such competing SPs. The CB decides the most appropriate
SP to host services based on the outcome of the game.
Each user application requests different VM instances, and we
assume that each application has the capability of executing
at every SP under the CB. The objective is to build a game
model where each user application can be represented as a
coalition of VMs executing at a SP. The CB aims to find a
coalition such that the aggregate user price is reduced. As a
consequence of this approach, VMs supporting an applications
might be hosted at different DCs at different geographical
locations. This might lead to the issues of VM dispersion.
\begin{table}[htbp!]
	\caption{REGION BASED VM PRICING FOR 3 SPs [5], [6], [7]}
	\begin{center}
		\begin{tabular}{c|c|c|c|c}
			\hline
			Type & Region 1 & Region 2 & Region 3 & Region 4\\
			Category & Central US & India west & Sydney & Tokyo\\
			(VM) & (\$/hr) & (\$/hr) & (\$/hr) & (\$/hr)\\
			\hline
		\multicolumn{5}{c}{	\textbf{Microsoft Azure}}\\
		\hline
		t1 (A1) & 0.012 & 0.014 & 0.015 & 0.015\\
		t2 (A2) & 0.048 & 0.054 & 0.059 & 0.059\\
		t3 (A3) & 0.023 & 0.027 & 0.029 & 0.029\\
		t4 (A4) & 0.094 & 0.11 & 0.12 & 0.12\\
		\hline
		\multicolumn{5}{c}{	\textbf{Amazon EC2}}\\
		\hline
		t1 (Small) & 0.023 & 0.0248 & 0.0292 & 0.0304\\
		t2 (Medium)) & 0.0464 & 0.0496 & 0.0292 & 0.0608\\
		t3 (Large) & 0.0928 & 0.0992 & 0.1168 & 0.1216\\
		t4 (2Exlarge) & 0.3712 & 0.3968 & 0.4672 & 0.4864\\
		\hline
		\multicolumn{5}{c}{	\textbf{Google Cloud}}\\
		\hline
		t1 (Standard-1) & 0.0475 & 0.057 & 0.0674 & 0.061\\
		t2 (Standard-2) & 0.0950 & 0.1141 & 0.1348 & 0.122\\
		t3 (Standard-3) & 0.19 & 0.2282 & 0.2697 & 0.244\\
		t4 (Standard-4) & 0.38 & 0.4564 & 0.5393 & 0.488\\
		\hline
		\end{tabular}
		\label{tab1}
	\end{center}
\end{table}
\newline
\tab VM dispersion is the act of executing VMs supporting the
same application at geographically distant DCs. This can lead
to degradation of the quality of service (QoS) in the form of
delayed response due to the communication overhead. This
is because links connecting DCs have limited capacity [8].
Hence, we also calculate the degradation in QoS, i.e., latency
involved for communicating VMs supporting an application.
Higher latency is intolerable; hence, we perform a trade-off
against the incurred cost. We solve this optimization problem
using the ant colony optimization (ACO) [9] based technique.
In a nutshell, the objectives of this paper can be highlighted
as follows.
\begin{itemize}
	\item We propose a non-cooperative game model among SPs
	to maximize their individual revenues.
	\item We develop a VM coalition architecture to find a min-
	imum cost coalition to reduce the aggregate user price
	to host user application in a multi-tier cloud platform.
	\item As VMs hosting the application suffer from VM disper-
	sion, we formulate the latency incurred in servicing a
	request if VMs are placed at different DCs under the
	same SP.
	\item We optimize the aggregate price and latency incurred to
	make the system real-time.
\end{itemize}
While the existing literature [10], [11], [12], [13] looks to
the problem from broker-perspective, we face the problem
from end-users’ requirements of optimizing the price as well
as the QoS. The proposed VM coalition architecture has
been implemented and simulated over CloudSim 3.0 simulator.
Results show that the proposed approach is able to reduce the
aggregate price but may sometimes lead to slightly higher la-
tency due to VM dispersion. In general, our approach for price
optimization is able to reduce the aggregate price of hosting
the resources by 35\% but leads to an increased latency of 15\%
in comparison to a random assignment technique which is
 primarily latency sensitive. However, we observe that the final
optimization based on ACO leads to an allocation vector that
balances both the price and latency in a pareto-optimal scale.
This shows the efficiency of the proposed solution which can
be applied on a variety of cloud applications over broker-based
multi-cloud federated platforms.
\newline
\tab The rest of the paper is organized as follows. Section II deals with the literature that we have reviewed. Section III
discusses the proposed model in detail. Section IV discusses
the simulation results, and finally in Section V, we draw the
conclusions and discuss the future works in this direction.
\section{RELATED WORK}
\tab A number of works in the literature has focused on various
aspects of cloud federation, multi-tier cloud applications and
broker based multi-cloud federated platforms. Samaan [14] has
proposed a novel model for capacity sharing in a federation
of cloud infrastructure service providers. They modeled the
interactions among the service providers as a repetitive game
of VM outsourcing with an option of offering all unused
capacity in spot market. In [15], Mashayekhy et al. have
proposed a cloud federation model based on game theory to
enhance provider’s dynamic resources to fulfill users demand.
They proposed a dynamic cooperation among the SPs to form
a federation in order to provide the requested resources to the
users, yielding the highest total profit as well as individual
profit for each participating service providers. On the other
hand, Pillai and Rao [16] have proposed a resource allocation
model based on game theory. They propose coalition formation
among machines in the cloud. A number of recent systems and
prototypes in the similar direction has also developed cloud
federation strategies for better resource utilization and request
satisfaction following the centralized broker based federation
architecture, such as OnApp [17], Arjuna’s Agility framework
[18] etc.
\newline
\tab The recent literature on multi-cloud coordination have
considered broker based platforms. Quarati \emph{et al}. [10] have
proposed a hybrid cloud broker, which implies that the cloud-
broker can allocate a service to its private resources or public
clouds. In fact, revenue of a broker consists of brokering
service price and resource provisioning price. If the service
is executed on its private resources, the broker revenue is the
sum of two parts; otherwise, the CB only gets the brokering
service price, and the resource provisioning price is paid to the
public clouds. Mei \emph{et al}. [11], have proposed a broker-centric
architecture to minimize the cost incurred by the short-term
users. The broker rents reserved instances that have discounted
price and rents them to the short-term users with smaller
billing time units (BTU) at a significantly low prices. The
broker can also gain profits by earning the differences in the
price of reserved and on-demand VMs. In [12], the authors
have proposed a new kind of broker for cloud computing,
whose business relies on outsourcing VMs to its customers.
More specifically, the broker owns a number of reserved
instances of different VMs from several cloud providers and
offers them to its customers in an on-demand basis. Larumbe
and Sanso [13] have proposed an energy-aware VM placing
broker to minimize operational expenditures while respecting
constraints on QoS, power consumption and CO 2 emissions.
\newline
\tab From the above discussion, we observe that the federated
cloud architecture and the multi-cloud broker based platforms
have been well investigated in the existing literature, and practical systems exist to cater such requirements for the end-
users for cost optimization purpose. However, the existing
studies mostly look into the profit maximization for the broker
while ensuring budgetary constraints from the end-users. This
paper looks the problem from the end-user point of view,
where the objective is to minimize the cost for the end-users
while ensuring the QoS for the running application. This also
caters to the dynamic distribution of VMs over geo-distributed
DCs from a service provider. We primarily look into the
following perspectives – (a) select the SP that can provide
minimum cost service; (b) select the DCs for the SP that can
ensure QoS of the running application while minimizing the
cost for VM deployment. We consider that the broker gets a
fix share of the cost by selecting the best (optimized in terms
of QoS and price) configurations for the end-user based on the
service availability at different SPs.

\section{PROPOSED MODEL}

\tab In this section we first formally define a multi-cloud SP
broker system. The overall architecture of a CB is depicted in
Fig. 2. The CB interconnects the SPs and the end-user. The
VM allocation request from the user is processed by the CB;
based on the output, the broker initiates the VM allocation to
a SP and returns back the VM handle to the end-user. The
allocation model proposed in the paper runs at the CB based
on the service request from the end-user and the resource
availability as well as pricing scheme for the SPs. Let B be a
\begin{figure}[htbp]
	\centerline{\includegraphics[height=40mm,width=90mm]{fig2.png}}
	\caption{Multi-cloud broker system}
	\label{fig}
\end{figure}
\newline
broker which can be defined as B = \{SP${_{1}}$, SP${_{2}}$ , · · ·,SP${_{l}}$ \} with \emph{l} SPs. Each SP has many DCs across the globe and
provides services in the form of VM instances. A SP SP$_i$
consists of multiple geographically distributed DCs and is
defined as SP$_i$ = \{DC$^1_i$ , DC$^2_i$ , · · · , DC$^p_i$ \}, where p is the
total number of data DCs under SP$_i$ . For different SPs, value
of p will be different which provides heterogeneity to our architecture.  In our model, we
consider different instances and their variable pricing (based
on their location) for different SPs. For example Table I shows
the variable pricing of Amazon EC2, Microsoft Azure and
Google for similar VM instances across different geographical
locations. The following assumptions is made for the proposed
multi-cloud SP broker system.
\begin{enumerate}
	\item Resource capacity of a host machine at each SP is
	known.
	\item Pricing of each VM instance for all SPs are pre-defined.
	\item For any multi-tier application requesting different VM
	instances, a coalition can be possible from each SP under
	cloud-broker jurisdiction.
\end{enumerate}
\subsection{Broker Queue}
\tab The request of a multi-tier application arrives at the broker.
The broker processes the request in a first-in-first-out (FIFO)
order. A broker queue consists of l parallel service facilities
where l is the total SPs associated with the broker. Let N be
the capacity of capacity of the broker queue. Let $\lambda$$_n$ be the arrival rate and $\mu$$_n$ be service rate of each
SP, where n is the number of request in any defined interval.
\newline
The Maximum size of broker queue is N-l. We can define 
$\lambda$$_n$ and $\mu$$_n$ as follows.
\begin{equation}
{\otherlambda} =\begin{cases}
\otherlambda,& \text{ } 0\leq n\leq N\\
0,              & \text{n $>$ N}
\end{cases}
\end{equation}
\begin{equation}
{\othermu} =\begin{cases}
n\othermu,& \text{ } 0\leq n\leq c\\
c\othermu,              & c\leq n\leq N
\end{cases}
\end{equation}
Using $\tau$ = $\lambda$ /$\mu$ the steady state probability P$_n$ is given by
\begin{equation}
P_n =\begin{cases}
\frac{\tau ^n}{n!}P_0,& \text{ } 0\leq n\leq c\\
\frac{\tau ^n}{c!c^{\text{n-c'}}}P'_0,              & c\leq n\leq N
\end{cases}
\end{equation}
where,
\begin{equation}
P_0 =\begin{cases}
( \sum_{n=1}^{c-1} \frac{\tau ^n}{n!}+ \frac{\tau(1-(\frac{\tau}{c})^{\text{N-c+1}})}{c!(1-\frac{\tau}{c})} )^{\text{-1}},& \frac{\tau}{c} \not= 1 \\
( \sum_{n=1}^{c-1}\frac{\tau ^n}{c!}+\frac{\tau}{c}(N-c+1))^{\text{-1}},              & \frac{\tau}{c} = 1
\end{cases}
\end{equation}
\subsection{Service Coalition Game for Profit Maximization}\label{AA}
A standard game comprises of all conceivable strategies of
every player and their corresponding payoffs [20], [21]. To get
a service request placed in its own DCs, a SP can impose some
strategy. More number of service requests a SP can allocate,
the more profit it can gain. We define the following terms
related to service coalition selection game.
\newline
\tab In a single broker multi-cloud SP scenario, SPs, users and
broker have their own individual preferences. SPs want to
host maximum applications which leads to maximize their
profit while balancing the overall investment. Users have
different QoS and price specifications.

\subsection{Problem Formulation}
We solve the above mentioned optimization problem using ant colony optimization (ACO) [24]. ACO is a popular
technique that has been used to solve different multi-objective
optimization problems in the literature [9], [25]. Any algorithm
based on ACO has to take care of the three things, viz., (i.)
pheromone update, (ii.) definition of pheromone and heuristic
information and (iii.) pheromone and heuristic information.
The solution is mainly based on ACS. The detailed steps is
shown in Algorithm 2 and 1. Ants deposit and can smell
pheromones, they tend to follow the path that has higher
such deposits. After evaluating the quality of food ants deposit pheromones as per their evaluation and help other ants
discover the food.


\section{RESULTS}
We have evaluated the proposed model using CloudSim
3.0 simulator, which is a standard simulator framework for
evaluating cloud computing framework. The proposed architecture and VM allocation strategy at the CB has been
implemented using CloudSim to test the system performance.
Various parameters used for simulation are summarized in
Table II. As mentioned in Section I, we have used four different VM instances and their respective pricing models
each from, Amazon EC2 [5], Microsoft Azure [6] and Google
Cloud [7].


\begin{table}[htbp!]
	\caption{SIMULATION PARAMETERS.}
	\begin{center}
		\begin{tabular}{|c|c|c|}
			\hline
		Sl. No. & Category & Value\\
			\hline
		1 & Number of cloudlets& 250 \\
			\hline
			2 & Number of processing elements per host (pes) & 16\\
			\hline
			3 & Input File size & 500 Bytes \\
			\hline
				4 & Output File Size & 500 Bytes \\
			
			\hline
			5 & Number of Hosts & 500  \\
			\hline
			6 & Number of APPS & 50 - 250 \\
			\hline
			
		\end{tabular}
		\label{tab1}
	\end{center}
\end{table}
\begin{figure}[htbp]
	\centerline{\includegraphics[height=40mm,width=90mm]{fig3.png}}
	\caption{Time (ms) vs Application processed}
	\label{fig}
\end{figure}
\tab Fig. 3 shows the time required to process the applications
over the proposed architecture. We observe that the processing
time (queuing time + service time) increases exponentially as
the number of application requests increase. This is because
of the increase in waiting time of applications in the queue
with increasing number of request arrivals. With this pattern
of application processing time, next we observe how many
applications are placed at different SPs with respect to the
total number of application processed under the two schemes
- the proposed optimization based on ACO (mentioned as
ACO in the graphs), and the baseline mechanism as mentioned
earlier (random). Fig 4 denotes the relationship between the
application placed and the application processed over the
three SPs. It can be observed from the plot that maximum
applications are hosted at SP$_1$ and SP$_2$. This can be attributed
to the fact that SP$_1$ and SP$_2$ charge less for different VM
instance types which can be observed from Table I. Moreover,
it can also be observed from Table I that the difference in price
for same VM instance type at different geographical location is
less for SP$_1$ and SP$_2$ in comparison with SP$_3$, hence, leading
to larger number of application being hosted at SP$_1$ and SP$_2$.

\section{CONCLUSION AND FUTURE WORK}
\tab In this paper, we proposed a centralized cloud-broker that
selects VMs on behalf of end-users such that the aggregate
price of hosting the VMs is minimum. As this approach may
lead to the issue of VM dispersion, we formulated the overall
problem as a bi-objective optimization problem and solved it
using ACO based framework. The strategic interaction among
the SPs in modelled as a non-cooperative game as every SP
tries to maximize its revenue by maximizing the number of
requests accepted. Results show that the proposed approach is
able to achieve proper trade-off between the cost of hosting
an application and its QoS. However, in this paper, the service
set coalition for hosting services is selected from a single SP.
These VMs are hosted at separate DCs but under the umbrella
of a single SP. As an immediate extension of this work, we
would like to construct the VM service set by including VMs
from different SPs, i.e., a federated architecture. The pricing
framework involved in such an architecture involves further
complexities and is left to be explored in the future.



\begin{thebibliography}{00}
\bibitem{b1} D. Huang, B. He, and C. Miao, “A survey of resource management in
multi-tier web applications,” IEEE Communications Surveys Tutorials,
vol. 16, no. 3, pp. 1574–1590, 2014.
\newline
\bibitem{b2} R. Gibbons, “A primer in game theory,” 1992.\newline
\bibitem{b3} R. S. Gibbons, “An introduction to applicable game theory,” 1997.\newline
\bibitem{b4} C. Blum, “Ant colony optimization: Introduction and recent trends,”
Physics of Life Reviews, vol. 2, no. 4, pp. 353 – 373, 2005.
\newline
\bibitem{b5}  Arjuna’s Agility framework. [Online]. Available:
\href{https://onapp.com/federation}{https://onapp.com/federation}
\newline
\bibitem{b6} F. Magoul`es, J. Pan, and F. Teng, Cloud computing: Data-intensive
computing and scheduling. CRC press, 2012, ch. Game theoretical
allocation in a cloud datacenter, pp. 41–62.
\newline
\bibitem{b6}Onapp-Federation. [Online]. Available: \url{https://onapp.com/federation}
\end{thebibliography}

\end{document}
